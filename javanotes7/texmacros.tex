% ================== Definition of \obeywhitespace, copied from eplain.tex ================
\def\makeactive#1{\catcode`#1 = \active \ignorespaces}%
\begingroup
\catcode`@=11
\gdef\@obeywhitespaceloop#1{\futurelet\next\@finishobeywhitespace}%
\gdef\@removebox{%
  \ifhmode
    \setbox0 = \lastbox
    \ifdim\wd0=\parindent
      \setbox2 = \hbox{\unhbox0}%
      \ifdim\wd2=0pt
        \ignorespaces
      \else
        \box2 % Put it back: it wasn't empty.
      \fi
    \else
       \box0 % Put it back: it wasn't the right width.
    \fi
  \fi
}%
\makeactive\^^M \makeactive\ % No spaces or ^^M's from here on.
\gdef\obeywhitespace{%
\tolerance=100%
\makeactive\^^M\def^^M{\par\futurelet\next\@finishobeyedreturn}%
\makeactive\ \let =\ %
\aftergroup\@removebox%
\futurelet\next\@finishobeywhitespace%
}%
\gdef\@finishobeywhitespace{{%
\ifx\next %
\aftergroup\@obeywhitespaceloop%
\else\ifx\next^^M%
\aftergroup\gobble%
\fi\fi}}%
\gdef\@finishobeyedreturn{%
\ifx\next^^M\vskip\blanklineskipamount\fi%
\indent%
}%
\endgroup
\newskip\blanklineskipamount
\blanklineskipamount = 0pt
\def\frac#1/#2{\leavevmode
   \kern.1em \raise .5ex \hbox{\the\scriptfont0 #1}%
   \kern-.1em $/$%
   \kern-.15em \lower .25ex \hbox{\the\scriptfont0 #2}%
}%
\newdimen\hruledefaultheight  \hruledefaultheight = 0.4pt
\newdimen\hruledefaultdepth   \hruledefaultdepth = 0.0pt
\newdimen\vruledefaultwidth   \vruledefaultwidth = 0.4pt
\def\ehrule{\hrule height\hruledefaultheight depth\hruledefaultdepth}%
\def\evrule{\vrule width\vruledefaultwidth}%
%================================================================================

\def\displaycode{\par\medbreak\begingroup\small\ttfamily\parindent=35pt\obeywhitespace}
\def\donedisplaycode{\par\endgroup\medbreak}
\blanklineskipamount=-5pt

\newcommand{\<}{\ifmmode<\else{\tt\char`<}\fi}
\renewcommand{\>}{\ifmmode>\else{\tt\char`>}\fi}
\renewcommand{\|}{\ifmmode|\else{\tt\char`|}\fi}
\renewcommand{\{}{{\tt\char`\{}}
\renewcommand{\}}{{\tt\char`\}}}
\renewcommand{\^}{{\tt\char`\^}}

\newcommand{\code}[1]{\texttt{#1}}
\newcommand{\ptype}[1]{\textbf{\textsf{#1}}}
\newcommand{\classname}[1]{\textsl{\textsf{#1}}}
\newcommand{\atype}[1]{\texttt{\textsf{#1}}}
\newcommand{\newword}[1]{\textbf{\textit{#1}}}
\newcommand{\codedef}[1]{\textbf{\texttt{#1}}}
\newcommand{\bnf}[1]{{$\langle$\textit{#1}$\rangle$}}
\newcommand{\newcode}[1]{{\itshape#1}}
\newcommand{\start}[1]{\textsc{#1}}
\newcommand{\sourceref}[1]{\textit{#1}}
\newcommand{\jarref}[1]{\textit{#1}}
\newcommand{\mybreak}{\par\smallbreak\centerline{\large{$*\ *\ *$}}\smallbreak\par}
\newcommand{\weblink}[2]{#2}
\newcommand{\1}{{\texttt{\char`\\}}}
\newcommand{\2}{\ifmmode\pi\else{$\pi$}\fi}
\newcommand{\3}{\ifmmode{\mathscr O}\else{$\mathscr O$}\fi}
\newcommand{\4}{\ifmmode\Theta\else{$\Theta$}\fi}
\newcommand{\5}{\ifmmode\Omega\else{$\Omega$}\fi}

\newcounter{exercisecounter}
\newcommand{\exercise}{\par\bigskip\stepcounter{exercisecounter}\noindent\llap{\bfseries\arabic{exercisecounter}.\ }\ignorespaces}
\newenvironment{exercises}
   {\setcounter{exercisecounter}{0}
   \newpage
   \section*{Exercises for Chapter \thechapter}\addcontentsline{toc}{section}{Exercises for Chapter \thechapter}\markright{\textsc{Exercises}}
   \leftskip=25pt}
   {}
   
\newcounter{quizcounter}
\newcommand{\quizquestion}{\par\bigskip\stepcounter{quizcounter}\noindent\llap{\bfseries\arabic{quizcounter}.\ }\ignorespaces}
\newenvironment{quiz}
   {\setcounter{quizcounter}{0}
   \newpage
   \section*{Quiz on Chapter \thechapter}\addcontentsline{toc}{section}{Quiz on Chapter \thechapter}\markright{\textsc{Quiz}}
   \leftskip=25pt}
   {}

\newcommand{\dumpfigure}[1]{\par\bigskip \setbox0=\hbox{#1}
\dimen1=\pagetotal \advance\dimen1by50pt
\ifdim \dimen1<\pagegoal
   \dimen0=\ht0 \advance\dimen0by\pagetotal
   \ifdim \dimen0>\pagegoal \vfil\goodbreak \fi 
\fi
\centerline{\box0}\bigbreak}


%\newcommand{\dumpfigure}[1]{\par\bigskip \setbox0=\hbox{#1} \dimen0=\ht0 \advance\dimen0by\pagetotal
%\ifdim \dimen0>\pagegoal \vfil\goodbreak \fi \centerline{\box0}\bigbreak}

\newcommand{\Item}[1]{\par\hangafter=0
                         \hangindent=25pt
                         \noindent\llap{#1}\ignorespaces}

\newcounter{mylistcounter}
\newcommand{\mynumberedlist}[1]{\par\smallskip\setcounter{mylistcounter}{0}#1\par\medskip}
\newcommand{\mynumbereditem}{\par\smallskip\stepcounter{mylistcounter}\Item{{\bfseries\arabic{mylistcounter}.}}\ }

\newcommand{\mylist}[1]{\par\smallskip #1\par\medskip}
\newcommand{\myitem}{\par\smallskip\Item{$\bullet\;$}}

\newcommand{\glossaryitem}[2]{{\hangafter=1 \hangindent=30pt \parindent=0pt \textbf{#1}. #2\par\smallskip}}

% used only in the non-linked PDF.  The first page of the linked PDF is \pageonelinked.
\newcommand{\pageone}{
  \begin{titlepage}
    \vglue 1.5 in
    \centerline{\Huge{Pengenalan Pemrograman Menggunakan Java}}
    \vskip 0.4 in
    \centerline{\LARGE{Versi 7.0, Agustus 2014}}
    \vskip 1.5 in
    \centerline{\LARGE{David J. Eck}}
    \vskip 0.2 in
    \centerline{\Large{Universitas Hobart dan William Smith}}
    \vskip 1.5 in
    \centerline{Ini merupakan versi PDF dari buku online gratis yang tersedia}
    \centerline{di http://math.hws.edu/javanotes/.  Situs web ini mencakup}
    \centerline{source code untuk semua contoh program, jawaban untuk kuis,}
    \centerline{dan diskusi serta solusi untuk latihan.}
    
    \newpage
    \vglue 4.8 in
 
    {\leftskip=0.9 in \rightskip=0.9 in plus 0.2 in minus 0.2 in
 
    \noindent\copyright 1996--2014, David J. Eck
    
    \bigskip
    
    \small{
    
    \noindent David J. Eck (eck@hws.edu)

    \noindent Departemen Matematika dan Ilmu Komputer

    \noindent Universitas Hobart dan William Smith

    \noindent Jenewa, NY\quad 14456}
    
    \bigskip
    \medskip
    
    \noindent  Buku ini dapat didistribusikan dalam bentuk nonmodifikasi untuk tujuan nonkomersil.  
    Versi modifikasi dapat dibuat dan distribusikan untuk tujuan nonkomersil
    dengan syarat hasil modifikasi didistribusikan dengan lisensi yang sama dengan
    aslinya.  Secara lebih spesifik:
    Karya ini berada dibawah lisensi Creative Commons Attribution-NonCommercial-ShareAlike 3.0 License.
    Untuk melihat salinan lisensi ini, kunjungi http://creativecommons.org/licenses/by-nc-sa/3.0/.
    Penggunaan lainnya memerlukan izin dari penulis.
    
    \bigskip
    
    \noindent Situs web buku ini adalah:\ \ \ http://math.hws.edu/javanotes
    
    }

  \end{titlepage}
}

\newcommand{\pageonelinked}{
  \begin{titlepage}
    \vglue 1.3 in
    \centerline{\Huge{Pengenalan Pemrograman Menggunakan Java}}
    \vskip 0.4 in
    \centerline{\LARGE{Versi 7.0, Agustus 2014}}
    \vskip 1.2 in
    \centerline{\LARGE{David J. Eck}}
    \vskip 0.2 in
    \centerline{\Large{Universitas Hobart dan William Smith}}
   
   \vskip 1.2 in
   \centerline{Ini merupakan versi PDF dari buku online gratis yang tersedia di}
   \centerline{\url{http://math.hws.edu/javanotes/}.  Versi PDF tidak mencakup}
   \centerline{berkas source code, solusi untuk latihan, atau jawaban untuk kuis, tetapi}
   \centerline{memiliki tautan eksternal ke sumber daya tersebut, yang ditunjukkan dengan tulisan berwarna biru.}
   \centerline{PDF ini juga memiliki tautan internal yang berwarna merah.  Tautan ini dapat}
   \centerline{digunakan oleh \textit{Acrobat Reader} dan beberapa prgram pembaca PDF lainnya.}
    
    \newpage
    \vglue 4.8 in
 
    {\leftskip=0.9 in \rightskip=0.9 in plus 0.2 in minus 0.2 in
 
    \noindent\copyright 1996--2014, David J. Eck
    
    \bigskip
    
    \small{
    
    \noindent David J. Eck (eck@hws.edu)

    \noindent Departemen Matematika dan Ilmu Komputer

    \noindent Universitas Hobart dan William Smith

    \noindent Jenewa, NY\quad 14456}
    
    \bigskip
    \medskip
    
    \noindent Buku ini dapat didistribusikan dalam bentuk nonmodifikasi untuk tujuan nonkomersil.  
    Versi modifikasi dapat dibuat dan distribusikan untuk tujuan nonkomersil
    dengan syarat hasil modifikasi didistribusikan dengan lisensi yang sama dengan
    aslinya.  Secara lebih spesifik:
    Karya ini berada dibawah lisensi Creative Commons Attribution-NonCommercial-ShareAlike 3.0 License.
    Untuk melihat salinan lisensi ini, kunjungi http://creativecommons.org/licenses/by-nc-sa/3.0/.
    Penggunaan lainnya memerlukan izin dari penulis.
    
    \bigskip
    
    \noindent Situs web buku ini adalah:\ \ \ http://math.hws.edu/javanotes
    
    }

  \end{titlepage}
}


\newcommand{\pageoneluluall}{
  \begin{titlepage}
    \vglue 1.5 in
    \centerline{\Huge{Pengenalan Pemrograman Menggunakan Java}}
    \vskip 0.4 in
    \centerline{\LARGE{Versi 7.0, Agustus 2014}}
    \vskip 1.5 in
    \centerline{\LARGE{David J. Eck}}
    \vskip 0.2 in
    \centerline{\Large{Universitas Hobart dan William Smith}}
    \vskip 1.5 in
    \centerline{Ini merupakan versi cetak dari textbook online gratis yang}
    \centerline{tersedia di http://math.hws.edu/javanotes/.  Situs web}
    \centerline{mencakpu source code untuk semua contoh program, jawaban untuk}
    \centerline{kusi, diskusi dan solusi untuk latihan, dan}
    \centerline{glosarium.}
    
    \newpage
    \vglue 4.8 in
 
    {\leftskip=0.9 in \rightskip=0.9 in plus 0.2 in minus 0.2 in
 
    \noindent\copyright 1996--2014, David J. Eck
    
    \bigskip
    
    \small{
    
    \noindent David J. Eck (eck@hws.edu)

    \noindent Departemen Matematika dan Ilmu Komputer

    \noindent Universitas Hobart dan William Smith

    \noindent Jenewa, NY\quad 14456}
    
    \bigskip
    \medskip
    
    \noindent  Buku ini dapat didistribusikan dalam bentuk nonmodifikasi untuk tujuan nonkomersil.  
    Versi modifikasi dapat dibuat dan distribusikan untuk tujuan nonkomersil
    dengan syarat hasil modifikasi didistribusikan dengan lisensi yang sama dengan
    aslinya.  Secara lebih spesifik:
    Karya ini berada dibawah lisensi Creative Commons Attribution-NonCommercial-ShareAlike 3.0 License.
    Untuk melihat salinan lisensi ini, kunjungi http://creativecommons.org/licenses/by-nc-sa/3.0/.
    Penggunaan lainnya memerlukan izin dari penulis.
    
    \bigskip
    
    \noindent Situs web buku ini adalah:\ \ \ http://math.hws.edu/javanotes
    
    }

  \end{titlepage}
}

\newcommand{\pageoneluluone}{
  \begin{titlepage}
    \vglue 1.2 in
    \centerline{\Huge{Pengenalan Pemrograman Menggunakan Java}}
    \vskip 0.3 in
    \centerline{\LARGE{Versi 7.0, Agustus 2014}}
    \vskip 0.1 in
    \centerline{(\textit{Versi 7.0.1, dengan sedikit perubahan, Agustus 2015})}
    \vskip 0.7 in
    \centerline{\Huge{Bagian 1}}
    \vskip 0.2 in
    \centerline{\Large{(\textit{Bab 1 sampai 7 dari buku lengkap.})}}
    \vskip 0.7 in
    \centerline{\LARGE{David J. Eck}}
    \vskip 0.2 in
    \centerline{\Large{Universitas Hobart dan William Smith}}
    \vskip 1.5 in
    \centerline{Ini merupakan setengah bagian pertama dari textbook online gratis yang}
    \centerline{tersedia di http://math.hws.edu/javanotes/.  Situs web}
    \centerline{mencakup source code untuk semua contoh program, jawaban untuk}
    \centerline{kuis, dan diskusi serta solusi untuk latihan.}
    
    \newpage
    \vglue 4.8 in
 
    {\leftskip=0.9 in \rightskip=0.9 in plus 0.2 in minus 0.2 in
 
    \noindent\copyright 1996--2015, David J. Eck
    
    \bigskip
    
    \small{
    
    \noindent David J. Eck (eck@hws.edu)

    \noindent Departemen Matematika dan Ilmu Komputer

    \noindent Universitas Hobart dan William Smith

    \noindent Jenewa, NY\quad 14456}
    
    \bigskip
    \medskip
    
    \noindent  Buku ini dapat didistribusikan dalam bentuk nonmodifikasi untuk tujuan nonkomersil.  
    Versi modifikasi dapat dibuat dan distribusikan untuk tujuan nonkomersil
    dengan syarat hasil modifikasi didistribusikan dengan lisensi yang sama dengan
    aslinya.  Secara lebih spesifik:
    Karya ini berada dibawah lisensi Creative Commons Attribution-NonCommercial-ShareAlike 3.0 License.
    Untuk melihat salinan lisensi ini, kunjungi http://creativecommons.org/licenses/by-nc-sa/3.0/.
    Penggunaan lainnya memerlukan izin dari penulis.
    
    \bigskip
    
    \noindent Situs web buku ini adalah:\ \ \ http://math.hws.edu/javanotes
    
    }

  \end{titlepage}
}

\newcommand{\pageonelulutwo}{
  \begin{titlepage}
    \vglue 1.2 in
    \centerline{\Huge{Pengenalan Pemrograman Menggunakan Java}}
    \vskip 0.3 in
    \centerline{\LARGE{Versi 7.0, Agustus 2014}}
    \vskip 0.1 in
    \centerline{(\textit{Versi 7.0.1, dengan sedikit perubahan, Agustus 2015})}
    \vskip 0.7 in
    \centerline{\Huge{Bagian 2}}
    \vskip 0.2 in
    \centerline{\Large{(\textit{Bab 8 sampai 13 dari buku lengkap.})}}
    \vskip 0.7 in
    \centerline{\LARGE{David J. Eck}}
    \vskip 0.2 in
    \centerline{\Large{Universitas Hobart dan William Smith}}
    \vskip 1.5 in
    \centerline{Ini merupakan setengah bagian kedua dari textbook online gratis yang}
    \centerline{tersedia di http://math.hws.edu/javanotes/.  Situs web}
    \centerline{mencakup source code untuk semua contoh program, jawaban untuk}
    \centerline{kuis, dan diskusi serta solusi untuk latihan.}
    
    \newpage
    \vglue 4.8 in
 
    {\leftskip=0.9 in \rightskip=0.9 in plus 0.2 in minus 0.2 in
 
    \noindent\copyright 1996--2015, David J. Eck
    
    \bigskip
    
    \small{
    
    \noindent David J. Eck (eck@hws.edu)

    \noindent Departemen Matematika dan Ilmu Komputer

    \noindent Universitas Hobart dan William Smith

    \noindent Jenewa, NY\quad 14456}
    
    \bigskip
    \medskip
    
    \noindent  Buku ini dapat didistribusikan dalam bentuk nonmodifikasi untuk tujuan nonkomersil.  
    Versi modifikasi dapat dibuat dan distribusikan untuk tujuan nonkomersil
    dengan syarat hasil modifikasi didistribusikan dengan lisensi yang sama dengan
    aslinya.  Secara lebih spesifik:
    Karya ini berada dibawah lisensi Creative Commons Attribution-NonCommercial-ShareAlike 3.0 License.
    Untuk melihat salinan lisensi ini, kunjungi http://creativecommons.org/licenses/by-nc-sa/3.0/.
    Penggunaan lainnya memerlukan izin dari penulis.
    
    \bigskip
    
    \noindent Situs web buku ini adalah:\ \ \ http://math.hws.edu/javanotes
    
    }

  \end{titlepage}
}



